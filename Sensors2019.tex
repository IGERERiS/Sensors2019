\documentclass[aspectratio=169]{beamer}
\PassOptionsToPackage{dvipsnames}{xcolor}
%\usetheme{Antibes}
\usepackage{graphicx}
\usepackage{tikz}
\graphicspath{{images/}} 
\definecolor{myaqua}{RGB}{27, 143, 150}
\pgfdeclareimage[width=\paperwidth]{mybackground}{Back_ground_first.pdf}
\pgfdeclareimage[width=\paperwidth]{background_all}{Back_ground.pdf}
\setbeamertemplate{itemize item}{\color{myaqua}$\blacksquare$}
\setbeamerfont{title}{size=\Large}

\setbeamercolor{frametitle}{fg = myaqua}
\setbeamerfont{frametitle}{size=\huge}
\setbeamerfont{author}{series=\bf, size=\small}
\setbeamertemplate{sidebar right}

\setbeamertemplate{title page}{

        \begin{picture}(0,0)

            \put(-28,-152){%
                \pgfuseimage{mybackground}
            }

            \put(0,-142){%
                \begin{minipage}[b][45mm][t]{226mm}
                    \usebeamerfont{title}{\inserttitle\par}
                \end{minipage}
            }

            \put(0,-200){%
                \begin{minipage}[b][45mm][t]{226mm}
                    \usebeamerfont{author}{\insertauthor\par}
                \end{minipage}
            }


            \end{picture}

    }
    


\addtobeamertemplate{frametitle}{\vspace*{0.2cm}\hspace*{0.6cm}}

\title[...]{Development of High Performance \\ Magnetic Sensors and Sensing Tools }
\author[Our authors]{Profs Xixiang Zhang and Tad W. Patzek, KAUST \\ Presenter Dr. Timofey Eltsov}


%\usebackgroundtemplate%
%{%
%    \includegraphics[width=\paperwidth,height=\paperheight]{Back_ground.pdf}
%    
%}

\usebackgroundtemplate{\includegraphics[width=\paperwidth,height=\paperheight]{image3.jpg}}



\begin{document}

    \begin{frame}[plain]
        \titlepage
    \end{frame}

    \section{Introduction}

\begin{frame}{Motivation}

\begin{columns}[c] 

\column{.7\textwidth}

    
\begin{small}
\begin{itemize}

\item Electrically resistive \textcolor{blue} {composite materials} are being introduced to the oil and gas industry
\item Composite tubulars are \textcolor{blue} {4 times lighter} than made of steel, withstand same pressure and load, resistant to corrosion and lasts more than twice longer
\item The use of EM transparent casing materials requires the development of appropriate geophysical methods
\item We use \textcolor{blue} {magnetic sensors} to verify quality of the cement behind casing and perform monitoring of oil field development
\item Can be used in horizontal and vertical wells

\end{itemize}
\end{small}

\column{.3\textwidth} % Left column and width
\includegraphics[scale=0.35]{pipe2.eps}
\centering
%\textit{Manifa offshore oilfield, operated by Saudi Aramco [Bloomberg]}

\vspace{0.5 cm}

\begin{tiny}
\textbf{Casing comparison}
\begin{table}[]
\begin{tabular}{|l|l|l|}
\hline
Parameter                                                                  & Steel & Fiberglass \\ \hline
Weight 1 m, kg                                                             & 9.5   & 3.1        \\ \hline
\begin{tabular}[c]{@{}l@{}}Roughness \\ coefficient, mm\end{tabular}       & 0.03  & 0.0015     \\ \hline
\begin{tabular}[c]{@{}l@{}}Breaking axial \\ tension load, kN\end{tabular} & 278   & 144-427    \\ \hline
Max temperature, $^o$C                                                              & 200  & 150         \\ \hline
%Service years                                                              & 1-10  & 20         \\ \hline
\end{tabular}
\end{table}
\end{tiny}

\end{columns}
 
\end{frame}


\begin{frame}{Development of magnetic sensing tools \\ for petroleum industry}

\begin{columns}[c] 

\column{.6\textwidth}

\textbf{Magnetic cement quality detection tool:}
\begin{itemize}

\item Numerical simulation of \textcolor{blue} {magnetic cement response} in the borehole conditions
\item Laboratory experiments on magnetic cement detection
\item \textcolor{blue} {Prototyping} of laboratory tool

\end{itemize}


\textbf{Monitoring of reservoir development:}
\begin{itemize}

\item Reservoir development \textcolor{blue} {monitoring} EM tool concept
\item Numerical simulation of EM response during oil reservoir development

\end{itemize}


\column{.38\textwidth} % Left column and width
\includegraphics[scale=0.0985]{iron_powder.eps}
\includegraphics[scale=0.1980]{sensor.eps}
\centering

\end{columns}
 
\end{frame}

% Chapter6.2.eps

\begin{frame}{Schematic representation of cemented well}
\centering
\includegraphics[scale=0.5]{scheme_cementing.eps}

\end{frame}

%
%\begin{frame}{Schematic representation of cemented well}
%\includegraphics[scale=0.75]{cementing.eps}
%\end{frame}

\begin{frame}{The Deepwater Horizon, April 20, 2010}
\begin{columns}[c] 



\column{.50\textwidth}
\centering
\includegraphics[scale=0.8]{book1.eps}

\column{.50\textwidth}
\centering
\begin{Large}
Estimated losses: \\ \textcolor{blue} {62 billion \$}
\end{Large}

\end{columns} 

\end{frame}

\begin{frame}{Magnetic cement quality detection}

\begin{columns}[c] 

\column{.7\textwidth} % Right column and width
\begin{block}{}
\begin{itemize}
	\item Borehole environment is magnetized  by \textcolor{blue} {a low frequency} induction tool (1-10 kHz)
	\item Measurements can be made by coils or sensors 
	\item Alternating magnetic field provides necessary  \textcolor{blue} {noise immunity} of logging measurements and excludes the influence of the geomagnetic field
	\item The proposed method can determine poor-quality cementation through a non-conductive casing
	\item 200 MHz induction tool can detect non hardened cement 
\end{itemize}
\end{block}

\column{.1\textwidth} % Left column and width
\includegraphics[scale=0.14]{tool_view.eps}
\end{columns}


\end{frame}




\begin{frame}
\frametitle{Magnetic susceptibility logging}
\begin{itemize}
\item Transmitter coil generates magnetic field, that produces eddy current in the formation 
\item The secondary magnetic field is registered by receiver coil
\item Only vertical component of magnetic field is considered - $H_z$
\end{itemize}
\begin{center}
\includegraphics[scale=0.62]{Borehole_big.eps}
\end{center}
\end{frame}
%
%------------------------------------------------
%
\begin{frame}
\frametitle{Magnetic susceptibility logging}
\begin{itemize}
\item Primary and secondary magnetic fields induce an \textcolor{blue} {electromotive force} in the receiver coil
\item Primary magnetic field is much stronger than secondary one
\item Magnetic susceptibility tool is \textcolor{blue} {calibrated} by response measurement in the air and its subtraction from the measured signal 
\item There is no secondary magnetic field in the air
\end{itemize}
\end{frame}
%
%------------------------------------------------
%
\begin{frame}
\frametitle{Tool response measurement}


\begin{columns}[c] % The "c" option specifies centered vertical alignment while the "t" option is used for top vertical alignment

\column{.4\textwidth} % Right column and width
{
\begin{equation}
H_{z,m} =\textcolor{red}{H_{z,p}} + \textcolor{green}{H_{z,s}}
\end{equation}
}

\begin{equation}
\textcolor{green}{H_{z,s}} = H_{z,m} - \textcolor{red}{H_{z,p}}
\end{equation}

\vspace{\baselineskip}
$H_{z,m}$ - measured magnetic field, \textcolor{red}{$H_{z,p}$} - primary magnetic field, \textcolor{green}{$H_{z,s}$} - secondary magnetic field


\column{.4\textwidth} % Left column and width

\includegraphics[scale=0.8]{coils.eps}
\end{columns}

\end{frame}

\begin{frame}
\frametitle{Magnetic cement logging}
Cavity filled with cement and 2 cm cement debonding can be detected 

\begin{minipage}[h]{0.21\linewidth}
\center{\includegraphics[width=1\linewidth]{Borehole_cavity.eps}} \\
\end{minipage}
\hfill
\begin{minipage}[h]{0.24\linewidth}
\center{\includegraphics[width=1\linewidth]{logging_cavity.eps}} \\
\end{minipage}
\hfill
\begin{minipage}[h]{0.21\linewidth}
\center{\includegraphics[width=1\linewidth]{Borehole_debonding.eps}} \\
\end{minipage}
\hfill
\begin{minipage}[h]{0.24\linewidth}
\center{\includegraphics[width=1\linewidth]{logging_debond.eps}} \\
\end{minipage}

\end{frame}

\begin{frame}
\frametitle{Cement hardening logging, apparent resistivity}


\begin{columns}[c] 

\column{.27\linewidth} 
\begin{small}

\textcolor{blue} {Apparent resistivity} is calculated from logging signals using homogeneous medium approximation\\
\vspace*{0.2cm}
Resistivity obtained from \textcolor{blue} {signal phase} is more sensitive to cement resistivity \\
\vspace*{0.2cm}
Singularities can be observed when cross different layers

\end{small}

\column{.18\linewidth} 
\includegraphics[scale=1.4]{cement_solid_model.eps}

\column{.18\linewidth} 
Phase
\includegraphics[scale=0.44]{cement_solid_logg_rapp_phs.eps}

\column{.18\linewidth} 
Amplitude
\includegraphics[scale=0.44]{cement_solid_logg_rapp.eps}

\end{columns}


\end{frame}

\begin{frame}{Prototyping of lab magnetic sensing tools}

\begin{columns}[c] 

\column{.3\textwidth}

    
\begin{itemize}
\item Model of the media was made from \textcolor{blue} {plexiglas}
\item We manufactured several prototypes of magnetic sensing tool
\item \textcolor{blue} {Single coil solenoids}, double coil induction tools with and without ferromagnetic core
%\item Measurements are complicated by EM noise in the lab
%%\item The most stable results we have with ferritic core when using coaxial cables 
\end{itemize}

\column{.68\textwidth} % Left column and width

%\includegraphics[scale=0.1]{prototypes.pdf}
\includegraphics[scale=0.43]{plex_model.eps}

\end{columns}



\end{frame}



\begin{frame}{Measurement of incomplete lift of the cement}

\begin{columns}[c] 

\column{.33\textwidth}

\centering
\includegraphics[scale=0.174]{solenoid_in_model}    
    
%\begin{itemize}
%\item We fill part of the model with sand with iron powder (cement)
%\item Measuring inductance change every cm we drag solenoid through the borehole
%\item Lack of iron powder can be seen by value of solenoid inductance
%\item Inhomogeneous distribution of the iron in sand can be detected with the proposed tool
%\end{itemize}


\column{.33\textwidth}
\centering
\includegraphics[scale=0.0428]{model_filled.pdf}

\column{.33\textwidth} % Left column and width
\centering
\includegraphics[scale=0.50]{incomplete_lift_log.eps}



\end{columns}

\end{frame}

\begin{frame}{Measurements of sand with iron powder mix magnetic properties}


 \begin{columns}[c] 

\column{.53\textwidth}

    
\begin{itemize}
\item Different \textcolor{blue} {mixtures} of sand and iron powder was prepared
\item Magnetic measurements were made using magnetic permeability meter 
\item Studied magnetic permeability with 1, 2...10\% of iron
\item Iron powder makes sand (cement) \textcolor{blue}{magnetic}
\end{itemize}

\column{.45\textwidth} % Left column and width

\includegraphics[scale=0.39]{sand_mag_data.eps}
\includegraphics[scale=0.31]{sand_mag.eps}

\end{columns}



\end{frame}

\begin{frame}{Key Accomplishments - magnetic cement}

\begin{Large}
\begin{itemize}

\item Inhomogeneities filled with cement are \textcolor{blue} {visible on the logs}
\item 200 MHz induction tool can be used for cement \textcolor{blue} {solidification detection}
\item The proposed method can determine poor-quality cementation through a non-conductive casing
\item Proposed technology was \textcolor{blue} {verified} by lab experiments
\item Two papers have been published

\end{itemize}
\end{Large}


\end{frame}


%\begin{frame}{Monitoring of the oil reservoir development}
%
%\begin{itemize}
%\item Resistivity of brine is very low comparing to resistivity of oil
%\item Measurement of electromagnetic fields in a developed field can help assess the situation and increase the reservoir operation time.
%\item To monitor reservoir development, reliable and accurate sensors are needed to measure electromagnetic fields
%\end{itemize}
%
%\end{frame}

\begin{frame}{Feasibility study: water coning in oil reservoir}

\begin{columns}[c] 

\column{.56\textwidth}

%\begin{itemize}
%\item Water-coning has been observed in those oil producing wells in which water encroaches from below and shuts-off some or most of the oil production
%\item Most current solutions are reactive rather than proactive
%\item We present a multi-physics approach to estimating encroachment
%of a watercone
%
%\end{itemize}

\centering
\includegraphics[scale=0.34]{oil_rig_wc.eps}

\column{.44\textwidth}

\centering
\includegraphics[scale=0.34]{oil_rig_wc2.eps}

\end{columns}

\end{frame}

%\begin{frame}{Detecting water coning using magnetic sensors}
%
%\begin{columns}[c] 
%
%\column{.56\textwidth}
%   
%\begin{itemize}
%\item Determining the distance to the water cone will help to change the production rate and extend the life of the well
%\item When the water cone reaches the well it is necessary to carry out expensive activities
%\item Our task is to determine the cone approximation using a powerful source and sensitive magnetic sensors
%\end{itemize}
%
%
%\column{.45\textwidth}
%
%\includegraphics[scale=0.34]{oil_rig_wc2.eps}
%
%
%
%\end{columns}
%
%
%\end{frame}

\begin{frame}{Feasibility study: detecting water coning using magnetic sensors}

\begin{columns}[c] 

\column{.6\textwidth}

    
\begin{itemize}
\item The studied medium is irradiated by an electromagnetic field 
\item \textcolor{blue} {Vertical or horizontal wells}
\item The approach of the water front changes the measured signal, it is monitored in real time
\item The reservoir brine salinity is 200000 ppm
\item Aquifer resistivity is 1.72 Ohm$\cdot$m, oil reservoir resistivity is 27.5 Ohm$\cdot$m
\end{itemize}


\column{.4\textwidth}

\includegraphics[scale=0.34]{rig_monitoring.eps}

\end{columns}

\end{frame}

\begin{frame}{Numerical simulation of water coning}
\begin{itemize}

\item The watercone development is captured using a commercial \textcolor{blue} {reservoir simulator}
\item The electromagnetic response to the watercone is captured using electromagnetic simulations
\item We consider a \textcolor{blue} {homogeneous oil reservoir} underlain by an aquifer
\item Parameters of the simulation are typical for \textcolor{blue} {Middle Eastern} reservoirs
\end{itemize}

	\begin{minipage}{0.34\linewidth}
		\textit{Schematic representation of waterconing occurred in thick oil reservoir and part of the mesh used in the simulation}
	\end{minipage}
	\begin{minipage}{0.32\linewidth}
		\center{\includegraphics[scale=0.28]{scheme_watercone.eps} }
	\end{minipage}
	\hfill
	\begin{minipage}{0.32\linewidth}
		\center{\includegraphics[scale=0.19]{mesh.eps} }
	\end{minipage}



\end{frame}


\begin{frame}{Stages of water cone growth}


  \begin{minipage}{\textwidth}

	\begin{minipage}[b]{0.44\textwidth}
		\centering


\begin{tabular}{lll}
\textbf{Stage \#} & \textbf{\begin{tabular}[c]{@{}l@{}}Time, \\ months\end{tabular}} & \textbf{\begin{tabular}[c]{@{}l@{}}Distance \\ to cone, m\end{tabular}} \\
0                 & 0                                                                & 40                                                                      \\
1                 & 17                                                               & 33.7                                                                    \\
2                 & 33                                                               & 26.2                                                                    \\
3                 & 49                                                               & 15.8                                                                    \\
4                 & 53                                                               & 11.2                                                                    \\
5                 & 55                                                               & 3.7                                                                     \\
6                 & 56                                                               & 0

\end{tabular}
\vspace{1.7cm}

%		\rule{6.4cm}{3.6cm}



	\end{minipage}
	\hfill
	\begin{minipage}[b]{0.52\textwidth}
		\centering
		\includegraphics[scale=0.21]{cone_stages2.eps}

\textit{Reservoir simulation results. Dashed lines shows location of water front at different times}


\end{minipage}
\end{minipage}

\end{frame}

\begin{frame}{Electromagnetic response}

	\centering
	\includegraphics[clip,width=0.85\linewidth]{phase_diff_cmg.eps}
	\\ Apparent resistivity at frequency = 1 kHz vs. distance to the cone. Dynamic saturation distribution is obtained using reservoir simulation. Five EM tools are considered

\end{frame}

\begin{frame}{Key Accomplishments}

\begin{LARGE}

\begin{itemize}

\item We have conducted \textcolor{blue} {reservoir} simulations followed by \textcolor{blue} {electromagnetic} simulations that use the computed fluid distributions
\item For the assumed reservoir conditions, the watercone can be identified \textcolor{blue} {20 meters} away from the well by high sensitive magnetic sensors
\item One paper has been submitted
\end{itemize}

\end{LARGE}

\end{frame}


%\begin{frame}{Opportunities and Challenges}
%
%\begin{itemize}
%\item Laboratory measurements are complicated by EM noise
%\item High sensitive magnetic sensors can be really useful in Oil and Gas exploration
%\item Industries need accurate and temperature and pressure resistant sensors
%\end{itemize}
%
%\end{frame}



\begin{frame}{Future Research Plans and Expected Outcomes}

\begin{LARGE}

\begin{itemize}
\item Laboratory measurements of cement \textcolor{blue} {solidification} induction responce
\item High performance \textcolor{blue} {sensors testing} on the models of real environment in the lab (received in early April 2019)
\item Numerical simulation of EM response during \textcolor{blue} {oil field development}
\end{itemize}

\end{LARGE}
\end{frame}

\begin{frame}{Published papers in the scope \\ of the Sensor Initiative}

1) T. Eltsov and T. W. Patzek (2018) Numerical simulation of the magnetic cement induction response in the borehole environment. SEG Technical Program Expanded Abstracts 2018: pp. 799-803. 

2) T. Eltsov, T. W. Patzek (2019) Beyond Steel Casing: Detecting Zonal Isolation in the Borehole Environment, SPE Middle East Oil and Gas Show and Conference, 18-21 March, Manama, Bahrain, SPE-195036-MS

3) T. Eltsov, V. Torrealba, H. Hoteit, T.W. Patzek (2019) Electromagnetic detection of the water cone growth during reservoir development, 81st EAGE Conference and Exhibition 2019 (Accepted)


\end{frame}


\end{document}